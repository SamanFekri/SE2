1. retrieving a number to line up: the main functionality of the application is the possibility of retrieving a number  which gives them their position in the queue. after then it will be easier for customers to access the supermarket without standing in a line for so long.  this way  forces people to first approach the building and then wait in close proximity (though not in a line) until their number is called.

2. generate QR codes:  this feature would be scanned upon entering the store, thus allowing store managers to monitor entrances.

3. estimation process:  There is a real risk that the approach does not
work in the case the customer arrives to the grocery store after his/her number is called, or too early, as in this case we would get back into a physical line situation. This implies that the system should provide customers with a reasonably precise estimation of the waiting time and should alert them taking into account the time they need to get to the shop from the place they currently are.

4. booking a visit: this function indicate that u can reserve a slot to go to supermarket, its almost like reserving a slot to go to museum or exhibition but they have slight differences like the duration that take to be in a supermarket which we are not able to estimate their time of shopping. so we have to mention a feature which request customer to indicate the approximate expected duration of the visit.

5. system for the customers analysis: there's another option that works for long-term customers in which the time spent by customers is analyzed and the system is gonna predict the average time for that specefic customer based on their pervious visits.

6. indicating the category of items: The application might also allow users to indicate, if not the exact list of items that they intend to
purchase, the categories of items that they intend to buy. This would allow the application to plan visits in a finer way, for example allowing more people in the store, if it knows that they are going to buy different things, hence they will occupy different spaces in the store when they visit (thus respecting the requirement that people keep enough distance between them).