In this section, we elaborate some decision we take.
\begin{itemize}
    \item \textbf{Flutter:} Flutter is a free, open-source mobile SDK that can be used to create native-looking Android and iOS apps from the same code base. Flutter helps app developers build cross platform apps faster by using a single programming language, Dart, an object-oriented, class defined programming language. We chose to use this Framework in order to have a mobile app that works both with iOS and Android, with performances similar to the native development for each platform, but only having to write the code once. In Flutter, every piece of the user interface is a widget: like text, buttons, check boxes, images. There are also ”container widgets” that contain other widgets. Widgets can be stateless, which are immutable, or stateful, which have a mutable state and are used when we are describing a part of the user interface that can change dynamically.\\
    Lastly we want to remember that Flutter is not the only framework that can be used to build cross-platform mobile apps, another option can be React Native, based on JavaScript. We decided to use Flutter because of it’s advantages in comparison with React Native and other UI software development kit, which are: better performances and increasing popularity due to it’s simplicity of develop.
    
    \item \textbf{Golang:} Go is a statically typed, compiled programming language. Go has the same performance as C, but it is much easier to maintain than Java. Without the need for a virtual machine, Go boasts easier maintenance and no warming up period. These and many other characteristics are what make Golang stand out from its competitors. 
    
    \item \textbf{Docker:} Containers work a little like VMs, but in a far more specific and granular way. They isolate a single application and its dependencies—all of the external software libraries the app requires to run—both from the underlying operating system and from other containers. All of the containerized apps share a single, common operating system (either Linux or Windows), but they are compartmentalized from one another and from the system at large.\\
    The benefits of docker:
    \begin{itemize}
        \item Docker enables more efficient use of system resources.
        \item Docker enables faster software delivery cycles.
        \item Docker enables application portability.
        \item Docker shines for microservices architecture
    \end{itemize}

\end{itemize}